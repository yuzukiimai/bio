\documentclass{jsarticle}
\usepackage[dvipdfmx]{graphicx}
\usepackage{subcaption}
\captionsetup[figure]{justification=centering}
\captionsetup[table]{justification=centering}
\usepackage[ipa]{pxchfon}

\begin{document}
\vspace*{-20mm}
{\LARGE 2023 年度 バイオ/メディカルロボティクス レポート問題}
\begin{flushright}
\large 20C1015 今井悠月
\end{flushright}
\vspace*{10mm}

\textbf{{[設問 1]}}\hspace*{1zw}2023 国際ロボット展に行き,最新のロボット事情に関して調査・研究をせよ.\\
\hspace*{5.7zw}具体的には,4 社以上の会社・ブース(大学等の研究機関を含む)を訪問し,担当者と\\
\hspace*{5.7zw}議論をし,議論の内容・そこから得たものなどに関して報告書としてまとめること.\\

\hspace*{5.7zw}2023年11月29日(水)から12月2日(土)までの期間に,東京ビックサイトで開催された
\hspace*{5.7zw}2023国際ロボット展を訪れ,最新技術の情報を収集するとともに自身の研究のために視察した.
\hspace*{5.7zw}訪れた会社・ブースと,そこでの議論およびそこから得たものに関して以下に述べる.\\

\vspace*{4mm}
%---------------------------------------------------------------------------------------------------------------------------
\hspace*{4.7zw}ブース:\underline{DENSO(「複数台ロボット最適経路計画」を活用した"ぶつからない"高速ワーク整列)}\\

\hspace*{4.7zw}概要:マニピュレータ同士(3台)が互いに干渉せず,最短の経路を生成可能である.\\
\hspace*{8.7zw}その際,周辺機器も認識し衝突を避ける.\\
\hspace*{8.7zw}これにより,経路を作成する必要がなくなるため,エンジニアの負荷を低減できる.\\
\hspace*{8.7zw}熟練者が作成した経路から,さらに23.9%のサイクルタイム短縮を実現.\\


\hspace*{4.7zw}議論の内容(質疑応答):

\begin{itemize}
  \addtolength{\itemindent}{5.4zw}
  \item [Q.]経路を作成するのが負担とあるが,そこまで大変なものなのか.\\
  \hspace*{5.5zw}私も,対向二輪型の自律移動ロボットの経路計画を扱っているが,指定した経由点を通る
  \hspace*{5.5zw}ように予め設定するだけなのでそこまで大変ではない.また,このような産業ロボットは
  \hspace*{5.5zw}工程が一定であり,予測不可能な事態が起こる可能性は低いため,危険を予測し回避する
  \hspace*{5.5zw}経路はエンジニア側で簡単に設定できるものではないのか.
  \vspace*{1zh}

  \item [A.]経路計画のプログラミング工数は非常に大きく,微調整に苦労する.この技術を使えば,
  \hspace*{5.5zw}AIにより複数台ロボットの経路を自動で生成・統合できる.よって従来,試行錯誤を重ね
  \hspace*{5.5zw}ていた経路調整が不要になり,短時間での設備立ち上げを実現できる.\\
  \vspace*{1zh}

  \item [Q.]熟練者が作成した経路から,さらに23.9%のサイクルタイム短縮を実現とあるが,熟練者
  \hspace*{5.5zw}であっても作成する経路は常に一定ではないはず.特に使用する環境が異なればその変化
  \hspace*{5.5zw}も大きくなるはず.この表現は不適切ではないか.
  \vspace*{1zh}

  \item [A.]おっしゃるとおり必ずしも作成する経路が一定とは限らない.これはあくまで一例である.
  \hspace*{5.5zw}非常に良くできた経路のうちの一つと比較した結果である.また,このシステムはまだ現
  \hspace*{5.5zw}場で取り入れられていないものであるため,実験環境は一定である.もちろん環境が変わ
  \hspace*{5.5zw}ればこの比率は変わるため,あくまで本システムを取り入れると,この程度変わるのだと
  \hspace*{5.5zw}いう目安の一つだと思ってもらいたい.

  \newpage
  \vspace*{-10zh}

  \item [Q.]これは最適な経路を計画するものということだが,既存の経路計画アルゴリズムとは何が
  \hspace*{5.5zw}違うのか.最短の経路を探索する例としては,ダイクストラ法やその発展形の A*アルゴリ
  \hspace*{5.5zw}ズム,価値反復などが挙げられる.わざわざ開発しなくてもこれらを使用すれば良い話な
  \hspace*{5.5zw}のではないか.
  \vspace*{1zh}

  \item [A.]たしかにこれらを用いてできないこともないが,これらと異なる点として,事前に周辺
  \hspace*{5.5zw}機器を含めた CADデータを含めていることが挙げられる.このデータを用いて,ロ
  \hspace*{5.5zw}ボットのそれぞれの始点,経由点,終点を設定すると自動的に最も短い経路が生成され
  \hspace*{5.5zw}るというものなので,よりシンプルな話で済むことと,計算速度の面でも開発した独自
  \hspace*{5.5zw}のアルゴリズムのほうが優れていると思われる.また,複数台のロボットが互いに干渉
  \hspace*{5.5zw}しないようにする必要があるため,CADデータを含んだアルゴリズムの方が都合が良い.\\
\end{itemize}

\hspace*{4.7zw}議論から得たもの:

\hspace*{5.7zw}私も移動ロボットの経路計画問題を勉強しているため興味があり,本ブースへ伺った.
\hspace*{6.7zw}私は移動ロボットの経路を作成する工程が負担に感じたことはなかったため,エンジニア
\hspace*{6.7zw}の負担が減るという点に疑問を抱いたが,それは既存の ROS パッケージを用いているから
\hspace*{6.7zw}であって,実際にフルスクラッチで経路計画を作成するのは手間なのだとわかった.また,
\hspace*{6.7zw}私が経路計画で用いている既存のアルゴリズムでは,複数台ロボットが干渉しないように
\hspace*{6.7zw}最短経路を探すのは難しいことも知った.そこで新たなアプローチとして,物体の大きさ
\hspace*{6.7zw}を予め教えておく方法があり,その方法がロボット同士の干渉を防ぐのに有効であることを
\hspace*{6.7zw}知った.さらに,定量的に有効性を示したデータであっても,今回のようにあくまで一例であ
\hspace*{6.7zw}る可能性がある.そのため,システムを買う立場としては過剰に信頼せず,疑ってみる姿勢が
\hspace*{6.7zw}必要であると考えるきっかけになった.\\

\vspace*{4mm}
% ------------------------------------------------------------------------------------------------------------------

\hspace*{4.7zw}ブース:\underline{RT(人型協働ロボット Foodly)}\\

\hspace*{4.7zw}概要:お弁当・惣菜工場の製造ラインで人と並んで盛り付け作業ができる人型協働ロボット.\\
\hspace*{8.7zw}頭部のカメラで不定形なバラ積み食材をひとつずつ認識可能.\\
\hspace*{8.7zw}多品種に対応可能で,1台のロボットに複数の容器・食材を登録可能.\\
\hspace*{8.7zw}小柄な成人サイズと非常にコンパクトな設計であることから,隣にいても圧迫感がない.\\
\hspace*{8.7zw}ロボットの導入により,髪の毛やまつ毛などの混入リスクを低減できる.\\
\hspace*{8.7zw}ROS・ROS 2に対応している.\\


\hspace*{4.7zw}議論の内容(質疑応答):

\begin{itemize}
  \addtolength{\itemindent}{5.4zw}
  \item [Q.]頻繁に食材を掴み損ねているが,これは本システムを導入する上で問題ではないのか.\\
  \hspace*{5.5zw}また,掴み損ねる原因はビジョン由来のものか.それとも腕の制御の部分や把持のパラ
  \hspace*{5.5zw}メータ関連によるものか.
  \vspace*{1zh}

  \item [A.]本システムは,あくまでも人と並んで作業を行わせるものであるため,問題ではない.
  \hspace*{5.5zw}ロボットだけで作業を行うとしたら問題であるが,製造ラインであるため,流れてきた
  \hspace*{5.5zw}際に人間がミスを発見できる.少しでも人間の作業の負担を低減することが目的である.
  \hspace*{5.5zw}掴み損ねる原因としては,ビジョンよりも制御の部分が大きい.パラメータは食材に合わ
  \hspace*{5.5zw}せて変更するものであるため,最適化されていない可能性がある.


  \newpage
  \vspace*{-10zh}

  \item [Q.]人間の隣で作業することを想定していると思うが,人間との接触のリスクはないのか.
  \vspace*{1zh}

  \item [A.]おっしゃるとおり接触のリスクはある.ただし,挟み込まれることを防止した構造であり,
  \hspace*{5.5zw}AI制御によって人が当たっていても安全に作業を継続できる.素材も金属ではないため,
  \hspace*{5.5zw}怪我のリスクは限りなく少ない.\\
  \vspace*{1zh}


  \item [Q.]ロボットの頭部に取り付けられたカメラは RealSense であるが,わざわざ RealSense を
  \hspace*{5.5zw}用いているということは Depth も測っているのか.また,Depth を測っているとしたら,
  \hspace*{5.5zw}その影響による通信速度の低下はないのか.私も,RealSense を使用したことがあるが,
  \hspace*{5.5zw}Depth を測ると非常に処理が重かった.処理を軽くするための工夫が施されているのか.
  \vspace*{1zh}

  \item [A.]おっしゃるとおり,Depth も測っている.ただし,デフォルトのパラメータでは処理が
  \hspace*{5.5zw}重くなるため,周辺のパラメータチューニングは行っている.具体的には,フレームレー
  \hspace*{5.5zw}トや解像度を変更している.また,深度データに対してはフィルタをかけることでノイ
  \hspace*{5.5zw}ズを低減している.さらに,RealSense の ROS パッケージでは,同期設定というものが
  \hspace*{5.5zw}あり,有効にするとRGBデータとDepth データを同じタイミングでまとめて取得可能.
  \hspace*{5.5zw}描画速度を上げる工夫としては,OpenCVからOpenGLに切り替えることが挙げられる.\\
  \vspace*{1zh}


  \item [Q.]様々な食材に対応しているとのことだが,それぞれ形状や特性が異なるため,把持するに
  \hspace*{5.5zw}は各種パラメータのチューニングが必要になるはずである.これはユーザが設定する必要
  \hspace*{5.5zw}があるのか.
  \vspace*{1zh}

  \item [A.]これに関しては,こちら側で調整する.よって,予めユーザ側に取り扱う食材を聞いて,対
  \hspace*{5.5zw}処する.ピッキングの可否はこちら側でテストを承っているため,気軽に相談してほしい.\\

  \vspace*{1zh}


  \item [Q.]同じ食材でも,形状や色が若干異なっていたり,使用する環境の照明条件も変化すると考
  \hspace*{5.5zw}えられるが,これらの影響で認識に失敗することはないのか.
  \vspace*{1zh}

  \item [A.]もちろん,容器や食材の学習は事前に必要である.なるべく多様なデータセットを作成し,
  \hspace*{5.5zw}学習させることで,照明条件の変化や食材のわずかな変化にも対応している.\\
\end{itemize}

\hspace*{4.7zw}議論から得たもの:

\hspace*{5.7zw}私も研究でカメラ画像を扱っており,機械学習を用いた物体の認識にも取り組んだことがある
\hspace*{6.7zw}ため,興味があり本ブースへ伺った.本製品は ROS 対応であり,私も研究では ROS を用い
\hspace*{6.7zw}ているため,勉強になることが多かった.特に RealSense の処理を軽くする工夫は,今後取り
\hspace*{6.7zw}入れるべきだと思った.また,私は画像処理ライブラリにOpenCVしか用いたことがないが,
\hspace*{6.7zw}処理速度の面では,OpenGLの方が優れていると学んだため,今後使用していきたい.

%-------------------------------------------------------------------------------------------------------------------------------

\newpage

\vspace*{-10zh}

\hspace*{4.7zw}ブース:\underline{TechShare(小型 4 足歩行ロボット Unitree Go2)}\\

\hspace*{4.7zw}概要:低価格の 2次開発が可能な小型 4足歩行ロボット.\\
\hspace*{8.7zw}Go1 から更に進化した実証実験のエントリーモデル.\\
\hspace*{8.7zw}開発環境は,ROS ベースの開発に対応.\\
\hspace*{8.7zw}豊富な I/O が利用できるドッキングステーションを搭載.\\
\hspace*{8.7zw}これにより,ユーザ独自のセンサやコンピュータなどを追加搭載することが可能.\\
\hspace*{8.7zw}オープンソフトウェア開発環境として,Unitree Legged SDK 開発環境が提供されている.\\


\hspace*{4.7zw}議論の内容(質疑応答):

\begin{itemize}
  \addtolength{\itemindent}{5.4zw}
  \item [Q.]前のモデルであるGo1から進化したとあるが,具体的な変更点はなにか.
  \vspace*{1zh}

  \item [A.]Go2 は Go1 に対して,歩行性能やバッテリーの駆動時間,センシング機能が大幅に進化
  \hspace*{5.5zw}している.具体的には,バッテリーの容量が Go1 では 6000 [mAh] だったのに対し,Go2
  \hspace*{5.5zw}では,15000 [mAh] と約2倍以上に向上している.歩行性能としては,段差乗り越え能力
  \hspace*{5.5zw}が 16[cm],最大登坂角度が40度を実現している.センシング機能としては,新たに 4D 
  \hspace*{5.5zw}LiDARを搭載している.\\
  \vspace*{1zh}

  \item [Q.]私が研究で扱っている自律移動ロボットにもLiDARが搭載されているため,LiDAR が 
  \hspace*{5.5zw}どのようなセンサであるかは知っているが,2D LiDAR と 3D LiDAR しか耳にした
  \hspace*{5.5zw}ことがない.4D LiDAR とは,どのような LiDAR なのか.2D,3D LiDAR と比べて
  \hspace*{5.5zw}優れている点はなにか.
  \vspace*{1zh}

  \item [A.]4D LiDAR は,Unitree社独自開発のものである.実態は3D LiDAR であり,あくま
  \hspace*{5.5zw}でも製品名で 4D LiDAR という名前を使用しているだけである.なぜ 4D という名前を
  \hspace*{5.5zw}付けたかは,360度半天球検知可能であることが関係している.この LiDAR は,Go2の
  \hspace*{5.5zw}頭の下部に搭載されており,従来の2D/3D LiDAR では検出できなかった近接の低い障害
  \hspace*{5.5zw}物や落下の可能性がある段差・穴などの検知が可能である.\\
  \vspace*{1zh}

  \item [Q.]Go2のデモンストレーションで,ロボット本体を押しても転倒していなかったが,これは
  \hspace*{5.5zw}機械学習もしくは強化学習で転倒を防いでいるのか.それとも別のアルゴリズムが組み込
  \hspace*{5.5zw}まれているのか.
  \hspace*{5.5zw}
  \vspace*{1zh}

  \item [A.]転倒の防止に関しては,機械学習などは一切用いていない.ロボットに搭載された IMU
  \hspace*{5.5zw}のデータに基づき行われる.IMU によって,加速度や角速度などの慣性データを測定し,
  \hspace*{5.5zw}これにより自身の動きや姿勢を把握し,姿勢を維持するために必要な力を出力している.\\
  \vspace*{1zh}

  \item [Q.]私は研究開発の一環で,自律移動ロボットのプロジェクトであるつくばチャレンジに
  \hspace*{5.5zw}参加しているが,本ロボットを使用しているチームを見かけた.本ロボットは,ユーザ
  \hspace*{5.5zw}に2次開発をしてもらうことを前提としているが,元が高性能であることから,追加する
  \hspace*{5.5zw}要素や技術が限られてくると考えられる.転倒しない,段差にも強い,障害物も検知可能
  \hspace*{5.5zw}と自律移動に必要な機能を網羅していることは,2次開発の面ではむしろ欠点ではないか.
  
  \newpage

  \vspace*{-10zh}

  \item [A.]これに関しては,Unitree社が提供している開発環境である Unitree Legged SDK で解決
  \hspace*{5.5zw}できる.この SDK 環境には,メーカーが提供する歩行や階段などの標準の運動パターン
  \hspace*{5.5zw}を活用するハイレベルのAPIの環境と,ユーザ独自の運動パターンを開発できるローレ
  \hspace*{5.5zw}ベルのAPIの環境の2種類が提供されているため,ユーザの用途や開発のニーズに合わ
  \hspace*{5.5zw}せて選択していただければ問題ない.\\  
\end{itemize}

\hspace*{4.7zw}議論から得たもの:

\hspace*{5.7zw}私も研究でカメラ画像を扱っており,機械学習を用いた物体の認識にも取り組んだことがある
\hspace*{6.7zw}ため,興味があり本ブースへ伺った.本製品は ROS 対応であり,私も研究では ROS を用い
\hspace*{6.7zw}ているため,勉強になることが多かった.特に RealSense の処理を軽くする工夫は,今後取り
\hspace*{6.7zw}入れるべきだと思った.

%-------------------------------------------------------------------------------------------------------------------------------
\newpage

\textbf{{[設問 2]}}\hspace*{1zw}未来(100 年後)の手術室はどうなっているか?\\
\hspace*{5.7zw}分かりやすい図を 1 枚描き,事例等を挙げながら具体的に説明せよ.\\


\newpage

\textbf{{[設問 3]}}\hspace*{1zw}人間の手と全く同じ様に動作する義手が開発できる技術をもっているとする.\\
\hspace*{5.7zw}(3-1) この義手に必要不可欠な機能,性能などを具体的に述べよ.\\

\hspace*{4.7zw}(3-2) この場合,健康な腕を切断して義手に変更する事も元の機能を取り戻す意味で\\
\hspace*{8.4zw}医療行為とすることができるはずである.\\
\hspace*{8.4zw}この行為の是非について,技術的・倫理的観点から考えて個人的な意見を述べよ.\\

\end{document}
