\documentclass{jsarticle}
\usepackage[dvipdfmx]{graphicx}
\usepackage{subcaption}
\captionsetup[figure]{justification=centering}
\captionsetup[table]{justification=centering}
\usepackage[ipa]{pxchfon}

\begin{document}
\vspace*{-30mm}
{\LARGE 2023 年度 バイオ/メディカルロボティクス レポート問題}
\begin{flushright}
\large 20C1015 今井悠月
\end{flushright}
\vspace*{10mm}

\textbf{{[設問 1]}}\hspace*{1zw}2023 国際ロボット展に行き,最新のロボット事情に関して調査・研究をせよ.\\
\hspace*{5.7zw}具体的には,4 社以上の会社・ブース(大学等の研究機関を含む)を訪問し,担当者と\\
\hspace*{5.7zw}議論をし,議論の内容・そこから得たものなどに関して報告書としてまとめること.\\

\hspace*{5.7zw}2023年11月29日(水)から12月2日(土)までの期間に,東京ビックサイトで開催された
\hspace*{5.7zw}2023国際ロボット展を訪れ,最新技術の情報を収集するとともに自身の研究のために視察した.
\hspace*{5.7zw}訪れた会社・ブースと,そこでの議論およびそこから得たものに関して以下に述べる.\\


\hspace*{4.7zw}ブース:DENSO(「複数台ロボット最適経路計画」を活用した"ぶつからない"高速ワーク整列)\\

\hspace*{4.7zw}概要:マニピュレータ同士(3台)が互いに干渉せず,最短の経路を生成可能である.\\
\hspace*{8.7zw}その際周辺機器も認識し衝突を避ける.\\
\hspace*{8.7zw}これにより,経路を作成する必要がなくなるため,エンジニアの負荷を低減できる.\\
\hspace*{8.7zw}熟練者が作成した経路から,さらに23.9%のサイクルタイム短縮を実現\\


\hspace*{4.7zw}議論の内容(質疑応答)

\begin{itemize}
  \addtolength{\itemindent}{5.4zw}
  \item [Q.]経路を作成するのが負担とあるが,そこまで大変なものなのか.\\
  \hspace*{5.5zw}私も,対向二輪型の自律移動ロボットの経路計画を扱っているが,指定した経由点を通る
  \hspace*{5.5zw}ように予め設定するだけなのでそこまで大変ではない.また,このような産業ロボットは
  \hspace*{5.5zw}工程が一定であり,予測不可能な事態が起こる可能性は低いため,危険を予測し回避する
  \hspace*{5.5zw}経路はエンジニア側で簡単に設定できるものではないのか.
  \vspace*{1zh}

  \item [A.]経路計画のプログラミング工数は非常に大きく,微調整に苦労する.この技術を使えば,
  \hspace*{5.5zw}AIにより複数台ロボットの経路を自動で生成・統合できる.よって従来,試行錯誤を重ね
  \hspace*{5.5zw}ていた経路調整が不要になり,短時間での設備立ち上げを実現できる.
  \hspace*{5.5zw}
\end{itemize}


\newpage



\textbf{{[設問 2]}}\hspace*{1zw}未来(100 年後)の手術室はどうなっているか?\\
\hspace*{5.7zw}分かりやすい図を 1 枚描き,事例等を挙げながら具体的に説明せよ.\\


\newpage

\textbf{{[設問 3]}}\hspace*{1zw}人間の手と全く同じ様に動作する義手が開発できる技術をもっているとする.\\
\hspace*{5.7zw}(3-1) この義手に必要不可欠な機能,性能などを具体的に述べよ.\\

\hspace*{4.7zw}(3-2) この場合,健康な腕を切断して義手に変更する事も元の機能を取り戻す意味で\\
\hspace*{8.4zw}医療行為とすることができるはずである.\\
\hspace*{8.4zw}この行為の是非について,技術的・倫理的観点から考えて個人的な意見を述べよ.\\

\end{document}
